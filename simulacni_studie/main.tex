
\documentclass[a4paper, 11pt]{article}\usepackage{times}
\usepackage{graphicx}
\usepackage[hyphens]{url}


\usepackage[czech]{babel}
\usepackage[utf8]{inputenc}

\begin{document}

\begin{titlepage}
    \begin{center}
    \begin{figure}[!h]
            \centering
            \includegraphics[scale=0.7]{logo.eps}
        \end{figure}        
    
        \vspace{100px}
    
        \huge{
            \textbf{
                Simulační studie\\ provozu skiareálu \\
                Deštné v Orlických horách\\
            }
        }
        \vspace{40px}
        \large
        \textbf{
            Projekt v předmetu\\
            Matematické modelování a simulace\\
        }
        \vfill
    \end{center}
        \Large{
            \hfill\\
            Daniel Žárský, xzarsk04\\
            Matěj Macek , xmacek27 \hfill \today
        }

\end{titlepage}


\clearpage
\thispagestyle{empty}
	\tableofcontents
\newpage
\pagenumbering{arabic}
\setcounter{page}{1}
\section{Úvod} % Matej 
% KRATKY POPIS STREDISKA 
%par slov o modelu 
% hlavne cile simulace 
\section{Popis střediska}
\subsection{Plány střediska}
% 2 planky z internetu
\subsection{Režim provozu}
% oteviraci hodiny, kdy se vlaky rozjedou, jak se doby lisi v zavislosti na obdobi, PERMANENTKY "
\subsection{Lanové dráhy}
% delka, doba vyjezdu, frakvence nabirani lidi, kapacita, kratky technicky popis (rozdil sedackova x kotva napr.)
\subsection{Sjezdovky}
%delka obtiznost, moznosti kam prejet 
\subsection{Násvštěvníci}
%chovani prichody rychost 
\section{Model}  % Dan

\subsection{Návrh modelu}

\subsection{Popis modelu}

\subsection{Implementace modelu}

\subsection{Testování modelu}

\section{Experimenty}

\section{Vyhodnocení experimentů} %dan + matej

\section{Závěr} % dan + matej

\section{Zdroje informaci}




\end{document}

